\documentclass[11pt,a4paper]{article}
\usepackage[utf8]{inputenc}
\usepackage[english]{babel}
\usepackage[T1]{fontenc}
\usepackage{amsmath}
\usepackage{amsfonts}
\usepackage{amssymb}
\usepackage{subcaption}
\usepackage{makeidx}
\usepackage{graphicx}
\usepackage{fourier}
\usepackage{listings}
\usepackage{color}
\usepackage{hyperref}
\usepackage[left=2cm,right=2cm,top=2cm,bottom=2cm]{geometry}
\author{Richard Saleem}
\title{Analytic quadratic integration over the two-dimensional brillouion zone, G Wiesenekker, G te Velde and EJBaerends}

\lstset{language=C++,
	keywordstyle=\bfseries\color{blue},
	commentstyle=\itshape\color{red},
	stringstyle=\color{green},
	identifierstyle=\bfseries,
	frame=single}
\begin{document}

\maketitle
\newpage
\tableofcontents
\newpage




\section{2D}
\begin{equation}
E(k_x , k_y) = \frac{\hbar^2}{2m} (k_x^2+k_y^2)
\end{equation}
The polar coordinate system  
\begin{align}
x = r cos(\theta)\\
y = r sin (\theta)\\
r = \sqrt{x^2 + y ^2}
\end{align}
\[
Jacobian = 
  \begin{vmatrix}
   \frac{\partial x}{\partial r}  &  \frac{\partial u}{\partial r} \\
   \frac{\partial x }{\partial \theta} & \frac{\partial y}{\partial \theta}
  \end{vmatrix}
 =    \begin{vmatrix}
   cos(\theta)  &  Sin(\theta) \\
   -rsin(\theta) & rcos(\theta)
  \end{vmatrix}
  = r\]
\begin{align}
r^2= (k_x^2+k_y^2) = \frac{2m E}{\hbar^2} \label{r^2}\\
\frac{\partial E}{\partial r} = \frac{\partial \frac{\hbar^2}{2m} r^2 }{\partial r} = r \frac{\hbar^2}{m}\\
dr = dE \frac{m}{\hbar^2 r}
\end{align}
\subsection{$f = 1$ case}
and the main integral
\begin{align}
D(E) &= \int \delta(E-e(k))dk \\
&= \int_0^1 \int_0^{2\pi}r\delta(E-e(k))d\theta dr\\
&= \int 2\pi r \frac{m}{\hbar^2 r}\delta(E-e(k))dE\\ 
&= \int \frac{m 2\pi}{\hbar ^2}\delta(E-e(k))dE & \textbf{where} \int f(E) \delta(E-e(k))dE  = f(e) \\
&= \frac{2\pi m}{\hbar^2}
\end{align}
\subsection{$f = r^2$ case}
the main integral
\begin{align}
D(E) &= \int r^2 \delta(E-e(k))dk \\
&= \int_0^1 \int_0^{2\pi}r^3\delta(E-e(k))d\theta dr\\
&= \int 2\pi r^3 \frac{m}{\hbar^2 r}\delta(E-e(k))dE\\ 
&= \int \frac{r^2 m 2\pi}{\hbar ^2}\delta(E-e(k))dE & \textbf{where} \int f(E) \delta(E-e(k))dE  = f(e) \\
&= \frac{r^2 2\pi m}{\hbar^2}\\
&= \Big(\frac{2 m}{\hbar^2}\Big)^2 \pi E  & \ref{r^2}
\end{align}
\section{The limits on the u-integration of the goniometric equations}
the parameters specifying the intersection of the surface energy and with the sides of the triangle \\
$n_{jx}+n_{jy} = c_j $    ref(41)
to introduce the triangle sides we use the the stander formula for a line function $y = mx + b$, where m is the slope and b is the y-intercept the value of y in $x =  0$. \\
and the point slope:  $ (y-y_1) = m (x-x_1)$ where $(x_1,y_1)$ is a point in the given line.
\begin{align}
y-y_1 &= m (x-x_1)\\
y-y_1 &= \frac{\partial y}{\partial x} (x-x_1) \\
\partial x  (y-y_1) & = \partial y (x-x_1)\\ 
\partial x y - \partial y x + \partial y x_1 - \partial x y_1 &= 0 \\
n_{jx}x+n_{jy}y &= c_j & \text{where  } n_{jx} =  - \partial y, n_{jy}= \partial x , c = \partial x y_1 - \partial y x_1
\end{align}

\subsection{Ellipse}
the ellipse requires the solution of the goniometric equation ref (42)
\begin{align}
n_{jx}\{(E-q_1)/q_4 \}^{1/2}cos(u) + n_{jy}\{(E-q_1)/q_6\} ^ {1/2} sin (u)&= c_j&\text{where  } (j = 1,2,3)\\ b cos (u) + a sin (u)  =  c_j \label{acosubsinuc}
\end{align}
\begin{align}
 t = tan ( \frac{u}{2})\\
 sin (u) = 2sin (\frac{u}{2})cos (\frac{u}{2}) = 2 \frac{sin( \frac{u}{2})}{cos (\frac{u}{2})}cos ^2 (\frac{u}{2}) = \frac{2 tan (\frac{u}{2})}{sec ^2 (\frac{u}{2})} = \frac{2t}{1+ t^2} \\
 cos (u) = cos^2 (\frac{u}{2}) - sin^2 (\frac{u}{2}) = \Big[ 1- \frac{sin^2 (\frac{u}{2})}{cos^2 (\frac{u}{2})}\Big]cos^2 (\frac{u}{2}) = \frac{1- tan^2 (\frac{u}{2})}{sec^2 (\frac{u}{2})}  =\frac{1-t^2}{1+t^2}\\
\end{align}
\begin{align}
a_j(\frac{2t}{1+t^2})+ b_j(\frac{1-t^2}{1+t^2}) &= c_j  &\ref{acosubsinuc}\\
a_j2t + b_j(1-t^2) = c_j (1+t^2)\\
(c_j+b_j)t^2 - a_j 2t +(c_j-b_j)= 0 \label{Quadratic equations}
\end{align}
solving the Quadratic equations $\ref{Quadratic equations}$ gives us
\begin{align}
t_j &= \frac{2a_j \pm \sqrt{{4a_j}^2-4(c_j-b_j)(b_j+c_j)}}{2(b_j+c_j)} \\
u_j &= 2tan ^{-1} (t_j) & \textbf{for } (j = 1,2,3)
\end{align}
which give two solutions for each j:
\begin{equation}
u = u_1,u_2,u_3,u_4,u_5,u_6
\end{equation}

\subsection{Hyperbola}
The hyperbola form need a solution for the quadratic equation: ref (43)
\begin{align}
n_{jx}u^2 -c_j u + n_{jy}\{(E-q_1)/q_5\} &= 0  &\text{where  } (j = 1,2,3)
\end{align}
\begin{align}
a_j'u^2 + b_j'u + c_j' = 0
\end{align}
We can find the solution for $u$ by using the quadratic formula:
\begin{align}
u_j = \frac{-b_j ' \pm \sqrt{b_j '^2 - 4a_j'c_j'} }{2a_j'}
\end{align}
which give 2 solutions for each j:
\begin{equation}
u = u_1,u_2,u_3,u_4,u_5,u_6
\end{equation}


\subsection{Parabola}
The Parabola form need a solution for the quadratic equation: ref (44)
\begin{align}
-n_{jy}(q_4 / q_3)u^2 + n_{jx}u + n_{jy} \{(E-q_1)/q3 \} - c_j &= 0  &\text{where  } (j = 1,2,3)
\end{align}
\begin{align}
a_j'u^2 + b_j'u + c_j' = 0
\end{align}
We can find the solution for $u$ by using the quadratic formula:
\begin{align}
u_j = \frac{-b_j ' \pm \sqrt{b_j '^2 - 4a_j'c_j'} }{2a_j'}
\end{align}
which give two solutions for each j:
\begin{equation}
u = u_1,u_2,u_3,u_4,u_5,u_6
\end{equation}

\subsection{Straight line}
The straight line form need a solution for the equation: ref (45)
\begin{align}
n_{jy}u + n_{jx} \{(E-q_1)/q2 \} - c_j &= 0  &\text{where  } (j = 1,2,3)
\end{align}
\begin{align}
u = \frac{ c_j - \{(E-q_1)/q2 \} }{n_{jy}}
\end{align}
which give one solutions for each j:
\begin{equation}
u = u_1,u_2,u_3
\end{equation}

\subsection{Degenerate}
The degenerate form $E_q (x,y)= q_1 + q_4 x^2 $ need a solution for the equation of the two parallel lines : ref (38,39,40)\\ 
\begin{align}
x = \pm \{ (e-q_1)/q_4 \}^2 
\end{align}
with the parametrsation: 
\begin{align}
x=& \pm (e-q_1)/ \{q_4(e-q_1) \}^{1/2} & y = u
\end{align}
The degenerate form need a solution of the equation ref (41) 
\begin{align}
n_{jx}x+n_{jy}y = c_j\\
n_{jx}(e-q_1)/ \{q_4(e-q_1) \}^{1/2} + n_{jy}u = c_j \\
u = \frac{c_j - n_{jx}(e-q_1)/ \{q_4(e-q_1) \}^{1/2} }{n_{jy}}
\end{align}

The Jacobian is:
\begin{align}
\bigg[\frac {\partial (x,y)}{\partial (e,u)}\bigg] = 1/2\{ q_4(e-q_1)\}^{1/2}
\end{align}
The integral become
\begin{align}
V_1  &=  \int_{e,u} \delta (E-e)   1/2\{ q_4(e-q_1)\}^{1/2} de du \qquad = \qquad 1/2\{ q_4(e-q_1)\}^{1/2} [u] \\
V_2  &=  \pm \  1/2\{ q_4(e-q_1)\}^{1/2}  (e-q_1)/ \{q_4(e-q_1) \}^{1/2}   [u] \\
V_3  &=  \pm \  1/2\{ q_4(e-q_1)\}^{1/2} [\frac {1}{2}u^2] \\
V_4  &=  \pm \  1/2\{ q_4(e-q_1)\}^{1/2}(e-q_1)^2/ \{q_4(e-q_1) \} [u] \\
V_5  &=  \pm \  1/2\{ q_4(e-q_1)\}^{1/2} (e-q_1)/ \{q_4(e-q_1) \}^{1/2}  [\frac {1}{2}u^2] \\
V_6  &=  \pm \  1/2\{ q_4(e-q_1)\}^{1/2} [\frac{1}{3}u^3] \\
\end{align}

which give one solutions for each j:
\begin{equation}
u = u_1,u_2,u_3
\end{equation}

\section{Checking if a point is inside a triangle}
This simple check can tell us if the curve lies inside the triangle. The test give an existed point in the triangle area only if :

 \begin{align}
p(x,y) = c_0 + (c_1 - c_0) * s + (c_2 - c_0) * t
\end{align}
$c_0,c_1,c_2$ is the three corners of the triangle\\
the point p is inside the triangle if and only if $ 0 <  S  < 1$ ,  $0 < t < 1 $  and $s+t< 1$.

\section{3D}
need to include the stander triangel in 2D and the weight consept i have them on paper, and need to adde the stander tettrahedral

\subsection{Standard tetrahedron}
Any tetrahedron with corners ({$c_1,c_2,c_3,c_4$}) can be transformed to the standard tetrahedron with corners (0,0,0), (1,0,0), (0,1,0) and (0,0,1) by the affine transformation.
$f_q (x,y,z) = p_1+ p_2x + p_3y+ p_4z +P_5x^2+p_6xy+p_7xz+p8y^2+ p6yz+ p_10 z^2$ \\
 \[
F =
  \begin{bmatrix}
  1 & 0 & 0 & 0 & 0 & 0 & 0 & 0 & 0 & 0 \\
  1 & 1 & 0 & 0 & 1 & 0 & 0 & 0 & 0 & 0 \\
  1 & 0 & 1 & 0 & 0 & 0 & 0 & 1 & 0 & 0 \\
 1 & 0 & 0 & 1 & 0 & 0 & 0 & 0 & 0 & 1 \\
 1 & 0.5 & 0 & 0 & 0.25 & 0 & 0 & 0 & 0 & 0 \\
 1 & 0 & 0.5 & 0& 0 & 0 & 0 & 0.25 & 0 & 0 \\
 1 & 0 & 0 & 0.5 & 0 & 0 & 0 & 0 & 0 & 0.25 \\
 1 & 0.5 & 0.5 & 0 & 0.25 & 0.25 & 0 & 0.25 & 0 & 0 \\
 1 & 0 & 0.5 & 0.5 & 0 & 0 & 0 & 0.25 & 0.25 & 0.25 \\
 1 & 0.5 & 0 & 0.5 & 0.25 & 0 & 0.25 & 0 & 0 & 0.25 \\
  \end{bmatrix}
\]

 \[
FI =
  \begin{bmatrix}
  1 & 0 & 0 & 0 & 0 & 0 & 0 & 0 & 0 & 0 \\
  -3 & -1 & 0 & 0 & 4 & 0 & 0 & 0 & 0 & 0 \\
  -3 & 0 & -1 & 0 & 0 & 4 & 0 & 0 & 0 & 0 \\
  -3 & 0 & 0  & -1& 0 & 0 & 4 & 0 & 0 & 0 \\
 2 & 2 & 0 & 0 & -4 & 0 & 0 & 0 & 0 & 0 \\
 4 & 0 & 0 & 0& -4 & -4 & 0 & 4 & 0 & 0 \\
 4 & 0 & 0 & 0 & -4 & 0 & -4 & 0 & 0 & 4 \\
 2 & 0 & 2 & 0 & 0 & -4 & 0 & 0 & 0 & 0 \\
 4 & 0 & 0 & 0 & 0 & -4 & -4 & 0 & 4 & 0 \\
 2 & 0 & 0 & 2 & 0 & 0 & -4 & 0 & 0 & 0 \\
  \end{bmatrix}
\]

 \subsection{3D-z}
we need to find the coordinates for each triangle for each Z-point we chose in our numerical integration.
if we chose z = 0 , we get our tetrahdral base. (0,0), (1,0) , (0,1).
parameterizing to z. z form 0 to 1:
(0,0),($1-z$,0),(0,$1-z$),($\frac{1-z}{2}$,0),(0,$\frac{1-z}{2}$),($\frac{1-z}{2}$,$\frac{1-z}{2}$)
\section{What is next?}
with this integration method we can get good but slow result but this will be enough to make a machine learning program which can calc much faster and better result with much lower k points. but first we need to a huge data set "learning set" .

\end{document}